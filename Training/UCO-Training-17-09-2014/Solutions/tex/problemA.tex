\chapter{Problem A: Hardwood Species}

En este problema hay que determinar, dadas unas especies de árboles, qué porcentaje representan del total. Este tipo de problemas de clasificación y agrupación de datos pueden resultar en \emph{time-outs} para los competidores nóveles sino escogen una estructura correcta. De hecho, el competidor tendrá problemas si quiere definir los datos en un simple vector, porque los índices no son enteros ($0,1,\dots,n$), por el contrario, los índices están dados por el nombre de la especie, de forma que podamos contar fácil y rápidamente las veces que aparece cada especie, por ejemplo:

\begin{center}
\begin{tabular}{c c c c c} 
	\hline	 
	\multicolumn{1}{|c}{1} &
	\multicolumn{1}{|c}{3} &
	\multicolumn{1}{|c}{2} &
	\multicolumn{1}{|c}{$\cdots$} &
	\multicolumn{1}{|c|}{$\# apariciones$} \\ \hline		
	\multicolumn{1}{c}{Ash} &
	\multicolumn{1}{c}{Gum} & 
	\multicolumn{1}{c}{Cherry} &
	\multicolumn{1}{c}{$\cdots$} &
	\multicolumn{1}{c}{Especie $n$} 	
\end{tabular}
\end{center}

Dicho esto, necesitamos una estructura en la que podamos definir el índice como \emph{string}, para lo cuál usaremos un \emph{map\textless string, double \textgreater trees }, el \emph{double} es para que al final realicemos realicemos la división que nos de el porcentaje sin necesidad de hacer un casting. Dicho porcentaje está dado por la siguiente fórmula: 

\begin{equation}
\% especie = \frac{trees[especie]}{total}
\end{equation}

En el ejemplo, trees[Ash] valdría 1, trees[Gum] valdría 3, y así sucesivamente. Para hallar el total basta con contar cuántos datos nos ingresan. \\

No sobra resaltar, que por el formato de la entrada se debe utilizar \emph{getline}, de forma que podamos saber cuando termina un caso de prueba, que para este problema está especificado con una línea vacía (que el \emph{cin} ignoraría).

\newpage

\section*{Código fuente}
\lstinputlisting[language=C++]{./source/Hardwood.cpp}



