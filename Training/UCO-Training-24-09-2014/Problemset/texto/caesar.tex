\begin{problema}{Caesar's Legions}{Standard}{Standard}{CodeForces}


Gaius Julius Caesar, a famous general, loved to line up his soldiers. Overall the army had $n_1$ footmen and $n_2$ horsemen. Caesar thought that an arrangement is not beautiful if somewhere in the line there are strictly more that $k_1$ footmen standing successively one after another, or there are strictly more than $k_2$ horsemen standing successively one after another. Find the number of beautiful arrangements of the soldiers. \\

Note that all $n_1 + n_2$ warriors should be present at each arrangement. All footmen are considered indistinguishable among themselves. Similarly, all horsemen are considered indistinguishable among themselves. \\


\InputFile

The only line contains four space-separated integers $n_1$, $n_2$, $k_1$, $k_2$ $(1 \leq n_1, n_2 \leq 100, 1 \leq k_1, k_2 \leq 10)$ which represent how many footmen and horsemen there are and the largest acceptable number of footmen and horsemen standing in succession, correspondingly.  \\ 


\OutputFile

Print the number of beautiful arrangements of the army modulo 100000000 $(10^8)$. That is, print the number of such ways to line up the soldiers, that no more than $k_1$ footmen stand successively, and no more than $k_2$ horsemen stand successively. \\



\Example

\input ejemplos/caesar.txt

Note: Let's mark a footman as 1, and a horseman as 2. \\

In the first sample the only beautiful line-up is: 121. \\

In the second sample 5 beautiful line-ups exist: 12122, 12212, 21212, 21221, 22121.\\

URL:\\ 
http://codeforces.com/problemset/problem/118/D


\end{problema}
