\begin{problema}{Software Allocation}{Standard}{Standard}{UVa}

A computing center has ten different computers (numbered 0 to 9) on which applications can run. The computers are not multi-tasking, so each machine can run only one application at any time. There are 26 applications, named $A$ to $Z$. Whether an application can run on a particular computer can be found in a job description (see below). 


Every morning, the users bring in their applications for that day. It is possible that two users bring in the same application; in that case two different, independent computers will be allocated for that application. 



A clerk collects the applications, and for each different application he makes a list of computers on which the application could run. Then, he assigns each application to a computer. Remember: the computers are \emph{not} multi-tasking, so each computer must handle at most one application in total. (An application takes a day to complete, so that sequencing i.e. one application after another on the same machine is not possible). 

A job description consists of 

\begin{enumerate}
\item one upper case letter $A\dots Z$, indicating the application.
\item one digit $0\dots 9$, indicating the number of users who brought in the application.
\item a blank (space character.)
\item one or more different digits $0\dots 9$, indicating the computers on which the application can run.
\item a terminating semicolon ``;''.
\item an end-of-line. 
\end{enumerate}



\InputFile

The input for your program is a textfile. For each day it contains one or more job descriptions, separated by a line containing only the end-of-line marker. The input file ends with the standard end-of-file marker. For each day your program determines whether an allocation of applications to computers can be done, and if so, generates a possible allocation. 


\OutputFile

The output is also a textfile. For each day it consists of one of the following: 

\begin{itemize}
\item ten characters from the set $A \dots Z , \_ $, indicating the applications allocated to computers 0 to 9 respectively if an allocation was possible. An underscore ``\_'' means that no application is allocated to the corresponding computer.

\item a single character ``!'', if no allocation was possible. 
\end{itemize}


\Example

\input ejemplos/software.txt


URL:\\ 
http://uva.onlinejudge.org/index.php? \\
option=onlinejudge\&page=show\_problem\&problem=195

\end{problema}
